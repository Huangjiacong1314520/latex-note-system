% main.tex - 使用示例
\documentclass[12pt]{article}

% 中文支持
\usepackage[UTF8]{ctex}
\usepackage{amsmath, amssymb}

% 导入笔记宏包
\usepackage{researchnotes}

% 如果想隐藏所有笔记,取消下面一行的注释
% \hidenotes

% 如果之前隐藏了笔记,想重新显示,使用
% \shownotes

\begin{document}

\section{示例章节}

这是正文内容。下面是一个研究笔记的例子。

\begin{researchnote}[author=JC, date=2025-10-16]
\textbf{包含关系:}
\[
Q_d \subset \{\tilde{\Delta}_d(j\omega) : \|\tilde{\Delta}_d(s)\|_\infty \leq 1\}
\]

但关键是:在计算最坏情况增益时,这个包含关系实际上是"充分"的。

\textbf{等价性定理:}

对于最坏情况增益计算,有:
\[
\sup_{\tilde{\Delta}_d \in \Delta_d} \sigma(\mathcal{F}_u(D(j\omega), \tilde{\Delta}_d(j\omega))) 
= \sup_{Q_d \in Q_d} \sigma(\mathcal{F}_u(D(j\omega), Q_d))
\]

这意味着:
\begin{itemize}
\item 虽然 $Q_d$ 是 $\tilde{\Delta}_d(j\omega)$ 的一个子集
\item 但在寻找最大增益时,我们不会丢失任何信息
\item 因为本质上这个 $Q_d$ 就是所有最坏增益对应的不确定性的集合
\end{itemize}
\end{researchnote}

medhighmedlow
\begin{researchnote}[date=2025-10-16]
    其实没什么使其
\end{researchnote}
\begin{questionnote}[author=JC, date=2025-10-16, class=]
如何找到这个最坏增益对应的不确定性的集合 $Q_d$?其实本质上也就是找 $Q_d$ 中的各个元素 $Q_j$
\end{questionnote}

\subsection{Example 2.24}

Consider the uncertain system
\begin{align*}
\tilde{G}_d(s,\Delta) &= (7G_1(s)(0.25+\delta_1) + 3G_2(s)(1-\delta_1) \\
&\quad + 3G_3(s)(0.75-\delta_1))(1+W(s)\Delta_1(s))
\end{align*}

with $G_1 = \frac{1}{(s+2)^2}$, $G_2 = \frac{10\pi s}{(s+200\pi)^2}$, and $G_3 = \frac{100\pi s}{(s+200\pi)^2}$, dynamic weight $W(s) = 0.2\frac{s+10\pi}{s+0.1\pi}$, and uncertain parameters $\delta_1 \in \mathbb{R}$ with $|\delta_1| \leq 1$, and $\Delta_1(s)$ with $\|\Delta_1\|_\infty \leq 1$.

% 更多示例笔记
\begin{researchnote}[author=张三, date=2025-10-18]
这是另一个研究笔记的例子,展示了如何使用不同的作者和日期。

可以包含:
\begin{itemize}
\item 数学公式
\item 列表
\item 各种文本格式
\end{itemize}
\end{researchnote}

\end{document}